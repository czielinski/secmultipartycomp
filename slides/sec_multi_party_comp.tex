%% LyX 2.1.3 created this file.  For more info, see http://www.lyx.org/.
%% Do not edit unless you really know what you are doing.
\documentclass[english]{beamer}
\usepackage{mathptmx}
\usepackage[T1]{fontenc}
\usepackage[latin9]{inputenc}
\usepackage{mathrsfs}
\usepackage{amsmath}
\usepackage{amssymb}
\usepackage{graphicx}

\makeatletter

%%%%%%%%%%%%%%%%%%%%%%%%%%%%%% LyX specific LaTeX commands.
%% Because html converters don't know tabularnewline
\providecommand{\tabularnewline}{\\}

%%%%%%%%%%%%%%%%%%%%%%%%%%%%%% Textclass specific LaTeX commands.
 % this default might be overridden by plain title style
 \newcommand\makebeamertitle{\frame{\maketitle}}%
 % (ERT) argument for the TOC
 \AtBeginDocument{%
   \let\origtableofcontents=\tableofcontents
   \def\tableofcontents{\@ifnextchar[{\origtableofcontents}{\gobbletableofcontents}}
   \def\gobbletableofcontents#1{\origtableofcontents}
 }

%%%%%%%%%%%%%%%%%%%%%%%%%%%%%% User specified LaTeX commands.
\usetheme{Antibes}
\usecolortheme{beaver}
\usefonttheme{professionalfonts}
\useinnertheme{circles}

\setbeamercovered{transparent}
\setbeamersize{text margin left=0.5cm}
\setbeamertemplate{footline}[frame number]
\setbeamertemplate{navigation symbols}{}

\usepackage{lmodern}
\usepackage[protrusion=true,expansion=true]{microtype}

\makeatother

\usepackage{babel}
\begin{document}

\title{Secure Multi-Party Computations}


\subtitle{An Introduction}


\author{Christian Zielinski}


\date{Last updated in 2015}

\makebeamertitle
\AtBeginSection[]{
  \frame<beamer>{ 
    \frametitle{Outline}   
    \tableofcontents[currentsection] 
  }
}
\begin{frame}{Outline}


\tableofcontents{}

\end{frame}

\section{Introduction}


\subsection{The problem}
\begin{frame}{Secure multi-party computations}

\begin{itemize}
\item Consider $n$ parties, with private inputs $x_{1},\ldots,x_{n}$
\item They want to compute a function $f\left(x_{1},\ldots,x_{n}\right)$
in a secure way
\item Security means here

\begin{itemize}
\item Privacy: The respective inputs remain private
\item Correctness: The output is guaranteed to be correct
\item Fairness: Each party learns the result
\end{itemize}
\item This should even hold when some parties try to cheat
\item \textbf{The following presentation is primarily based on Ref.~\cite{CDN09,HM97}}
\end{itemize}
\end{frame}

\begin{frame}{Questions at hand}

\begin{itemize}
\item How to carry out computations without revealing the inputs?
\item How to deal with cheating (corrupted) parties?
\item How to define security formally?
\item What is the upper limit of corrupted parties allowed?
\item How does this bound depend on the assumption made about the attacker?
\end{itemize}
\end{frame}

\subsection{Motivation and applications}
\begin{frame}{Motivation and applications}

\begin{itemize}
\item Multi-party computations (\textbf{MPC}s) have a wide\\
range of applications
\item Auctions

\begin{itemize}
\item Several parties are bidding for a product
\item Winning party and maximum bid should be determined, without revealing
bids of other parties
\end{itemize}
\item Electronic voting schemes

\begin{itemize}
\item Each party votes for a candidate
\item Only the result is made public, the votes remain secret
\end{itemize}
\item Yao's Millionaires' Problem, c.f.~Ref.~\cite{Yao82}

\begin{itemize}
\item A group of millionaires wants to find out who is the richest
\item Nobody wants to reveal how wealthy they are
\end{itemize}
\end{itemize}
\end{frame}

\section{Basic Terminology}


\subsection{Corrupted parties}
\begin{frame}{Adversaries}

\begin{itemize}
\item To discuss secure MPCs, we have to define security
\item Hence we have to make assumptions about cheating parties
\item Typically one models them by considering an \textbf{adversary}
\item This adversary can take over (\textbf{corrupt}) certain subsets of
parties
\item We assume one adversary, assuming the worst-case scenario of coordinated
corrupted parties (\textbf{monolithic adversary})
\item We assume that at the beginning of the protocol honest (i.e.~not
corrupted) parties do not know which parties are corrupted
\end{itemize}
\end{frame}

\begin{frame}{Passive and active security}


We distinguish two cases of corruption
\begin{definitions}
\textbf{Passive corruption}:
\begin{itemize}
\item Adversary has \textit{full information} of corrupted parties
\item However, corrupted parties still follow the protocol
\end{itemize}
\textbf{Active corruption}:
\begin{itemize}
\item Adversary has \textit{full control} over the corrupted parties
\item Might deviate from the protocol to obtain sensitive data
\end{itemize}
\end{definitions}

\end{frame}

\begin{frame}{Communication channels}


Parties have to communicate and coordinate
\begin{definitions}
The \textbf{information-theoretic model}:
\begin{itemize}
\item All parties have pairwise secure channels
\item Adversary has no access to messages sent between honest parties
\end{itemize}
The \textbf{cryptographic model}:
\begin{itemize}
\item The adversary has access to all messages sent
\item Messages cannot be altered, i.e.~the communication channel is authenticated
\end{itemize}
\end{definitions}


Sometimes we take a \textbf{broadcast} channel into account:
\begin{itemize}
\item All honest parties are assumed to receive the message\\
(\textbf{consensus broadcast})
\end{itemize}
\end{frame}

\begin{frame}{$\mathcal{A}$-adversaries}

\begin{itemize}
\item We define an \textbf{adversary structure} $\mathcal{A}\subset\mathcal{P}\left(P\right)$
as a family of subsets of the parties $P$
\item An \textbf{$\mathcal{A}$-adversary} can only corrupt subsets of parties
in $\mathcal{A}$

\begin{itemize}
\item $\mathcal{A}$ is monotone
\end{itemize}
\item Typically we consider \textbf{threshold adversaries}, i.e.~$\mathcal{A}$
contains all subsets of up to some cardinality $t$
\item An \textbf{adaptive} \textbf{$\mathcal{A}$-}adversary can corrupt
a new party during execution, if the total set is in \textbf{$\mathcal{A}$}
(otherwise call it \textbf{static})
\end{itemize}
\end{frame}

\subsection{Definitions of security}
\begin{frame}{Security in an ideal world}

\begin{itemize}
\item How to define security in general?
\item Here we introduce the concept of an \textbf{ideal world}
\item A protocol is secure if an adversary does not learn more in the \textbf{real
world} about the computations then in the ideal case
\item More formal: Consider a function $f$, which should be securely evaluated
in a MPC setting
\item We introduce an ideal functional $\textrm{\ensuremath{\mathcal{F}_{\mathrm{SFE}}^{f}}}$,
which is \textbf{incorruptible} and \textbf{leaks no private information}
\item Then the MPC problem reduces to the parties securely sending their
inputs to $\textrm{\ensuremath{\mathcal{F}_{\mathrm{SFE}}^{f}}}$
and receive the final result
\end{itemize}
\end{frame}

\begin{frame}{Secure implementations}

\begin{itemize}
\item In the real world $\textrm{\ensuremath{\mathcal{F}_{\mathrm{SFE}}^{f}}}$
is implemented by a protocol $\pi_{\mathrm{SFE}}^{f}$
\item We call $\pi_{\mathrm{SFE}}^{f}$ a \textbf{secure implementation}
of $\textrm{\ensuremath{\mathcal{F}_{\mathrm{SFE}}^{f}}}$ if an adversary
is unable to learn more about the computations than in the ideal world
(without help of trusted parties)
\item More formally assume an $\mathcal{A}$-adversary and let
\[
\mathfrak{I}=\mathrm{IDEAL}_{\mathcal{A}}\left(\textrm{\ensuremath{\mathcal{F}_{\mathrm{SFE}}^{f}}}\right)
\]
denote what an adversary learns in the ideal world
\item Similarly define
\[
\mathfrak{R}=\mathrm{REAL}_{\mathcal{A}}\left(\pi_{\mathrm{SFE}}^{f}\right)
\]
for the execution of the protocol in the real world
\end{itemize}
\end{frame}

\begin{frame}{Degrees of security}

\begin{itemize}
\item With $\mathrm{SIM}\left(\mathfrak{I}\right)$ we denote a \textbf{simulated
protocol} using only the information of $\mathfrak{I}$, which we
can compare with the real world protocol \textrm{$\mathfrak{R}$}
\item If $\mathfrak{R}$ contains no more information than $\mathrm{SIM}\left(\mathfrak{I}\right)$:

\begin{itemize}
\item $\pi_{\mathrm{SFE}}^{f}$ is a \textbf{perfect secure implementation}
\item No unwanted information leaks
\end{itemize}
\item If $\mathfrak{R}$ only contains additional statistical deviations
from $\mathrm{SIM}\left(\mathfrak{I}\right)$:

\begin{itemize}
\item $\pi_{\mathrm{SFE}}^{f}$ is a \textbf{statistically secure implementation}
\end{itemize}
\item If $\mathfrak{R}$ is only computationally indistinguishable from
$\mathrm{SIM}\left(\mathfrak{I}\right)$:

\begin{itemize}
\item $\pi_{\mathrm{SFE}}^{f}$ is a \textbf{computationally secure implementation}
\item Adversary cannot distinguish due to computational bounds
\end{itemize}
\end{itemize}
\end{frame}

\section{Protocols for Multi-Party Computations}


\subsection{Overview}
\begin{frame}{Threshold adversaries}

\begin{itemize}
\item We focus on threshold adversaries, i.e.~the adversary can corrupt
any set of parties up to cardinality $t$
\item In the information-theoretic with adaptive adversaries we have the
following results:
\begin{table}
\begin{centering}
\begin{tabular}{|c|c|c|c|}
\hline 
 & Passive & Active w/ BC & Active w/o BC\tabularnewline
\hline 
\hline 
Perfect & $n/2$ & $n/3$ & $n/3$\tabularnewline
\hline 
Statistical & $n/2$ & $n/2$ & $n/3$\tabularnewline
\hline 
Computational & $n$ & $n/2$ & $n/2$\tabularnewline
\hline 
\end{tabular}
\par\end{centering}

\protect\caption{Maximal obtainable threshold $t$ with $n$ parties (taken from Ref.~\cite{CDN09})}
\end{table}

\item Here we do not discuss general $\mathcal{A}$-adversaries, see Ref.~\cite{CDN09}
\end{itemize}
\end{frame}

\subsection{Passive secure protocol}
\begin{frame}{Perfect security with passive adversary}

\begin{itemize}
\item Assume $n$ parties and a passive threshold adversary\\
with threshold $t$
\item We construct a perfectly secure protocol in the information-model
theoretic for $t<n/2$
\item We employ Shamir's $\left(t+1,n\right)$-scheme, calculating\\
in a finite field $\mathbb{F}$
\item Assume parties agreed to calculate $s'=\mathfrak{O}\left(s\right)$
with secret $s$

\begin{itemize}
\item Secret $s$ has been securely shared, so that party $i$ has share
$s_{i}$
\item Carry out operations $s'_{i}=\mathfrak{O}_{i}\left(s_{i}\right)$
\item Shares $\left\{ s'_{i}\right\} $ allow to uniquely reconstruct $s'$
by $t+1$ parties
\end{itemize}
\end{itemize}
\end{frame}

\begin{frame}{Recap: Shamir's scheme}

\begin{itemize}
\item Assume $n$ parties and threshold $1\leq t\leq n$
\item Take a finite field $\mathbb{F}$ with $\left|\mathbb{F}\right|\geq n+1$
\item Let $s\in\mathbb{F}$ be the secret and define distinct elements $P_{1},\ldots,P_{n}\in\mathbb{F}\backslash\left\{ 0\right\} $ 
\item Sample a random polynomial $p$ with $\deg p\leq t-1$ and $p\left(0\right)=s$
\item Protocol:

\begin{itemize}
\item Distribution phase: dealer shares $s_{i}=p\left(P_{i}\right)$ privately
with party $i$
\item Reconstruction phase: $\geq t$ parties can reconstruct $p\left(x\right)$\\
(and hence $p\left(0\right)=s$)
\end{itemize}
\end{itemize}
\end{frame}

\begin{frame}{Recap: Addition}

\begin{itemize}
\item Assume $P_{i}$ has share $a_{i}$ and $b_{i}$
\item Assume $a$ and $b$ have been shared with (random) polynomials $p_{a}$
and $p_{b}$ of degree $\leq t$
\item We want to securely evaluate $c=a+b$

\begin{itemize}
\item Each party adds $c_{i}=a_{i}+b_{i}$ locally
\item The $\left\{ c_{i}\right\} $ uniquely determine the polynomial $p_{c}=p_{a}+p_{b}$
\item Polynomial $p_{c}$ encodes the result as $p_{c}\left(0\right)=p_{a}\left(0\right)+p_{b}\left(0\right)=a+b=c$
\end{itemize}
\item As $\deg p_{c}\leq t$ we find that $t+1$ parties can reconstruct
$c$
\item In the special case of adding a (public) constant $k$ party $i$
just calculates $s'_{i}=s_{i}+k$
\end{itemize}
\end{frame}

\begin{frame}{Recap: Multiplication}

\begin{itemize}
\item The case of multiplying by a (public) constant $k$ is similar
\item To securely evaluate $s'=k\cdot s$, every party calculates $s'_{i}=s_{i}\cdot k$
\item Shares $\left\{ s'_{i}\right\} $ determine polynomial $p_{s'}=k\cdot p_{s}$,
which encodes $p_{s'}\left(0\right)=k\cdot p_{s}\left(0\right)=k\cdot s=s'$
\item What about the general case of $c=a\cdot b$ with $a$ and $b$ secretly
shared?
\item Every party can calculate $a_{i}\cdot b_{i}$

\begin{itemize}
\item Uniquely determines the polynomial $p_{c}=p_{a}\cdot p_{b}$
\item Decodes the result as $p_{c}\left(0\right)=p_{a}\left(0\right)\cdot p_{b}\left(0\right)=a\cdot b=c$
\item \textbf{But is of degree $\mathbf{deg\:p_{c}\leq2t}$!}
\end{itemize}
\end{itemize}
\end{frame}

\begin{frame}{Secure degree reduction (I)}

\begin{itemize}
\item As $t<n/2$ we can at least uniquely define $p_{c}$
\item Now \textbf{securely reduce the degree} of $p_{c}$, so that $\deg p_{c}\leq t$
\item First observe by means of Lagrange interpolation
\[
a\cdot b=\sum_{1\leq i\leq n}\underbrace{\left[\frac{\prod_{1\leq j\leq n}^{j\neq i}\left(-P_{j}\right)}{\prod_{1\leq j\leq n}^{j\neq i}\left(P_{i}-P_{j}\right)}\right]}_{\equiv r_{i}}a_{i}\cdot b_{i}
\]

\item Hence we have a linear combination of the result $c$ in terms of
shares $\left\{ a_{i}\cdot b_{i}\right\} $ at hand
\item The \textbf{recombination vector} $r_{1},\ldots,r_{n}$ can be calculated
from public information
\end{itemize}
\end{frame}

\begin{frame}{Secure degree reduction (II)}

\begin{itemize}
\item In the next step each party acts as a dealer and \textbf{re-shares}
their share $a_{i}\cdot b_{i}$ using a polynomial $\mathfrak{c}_{i}$
of $\deg\mathfrak{c}_{i}\leq t$
\item This results in party $i$ having shares $u_{ji}$ ($j=1,\ldots,n$)
\item We can then consider the polynomial
\[
\mathfrak{c}=\sum_{1\leq i\leq n}r_{i}\cdot\mathfrak{c}_{i}
\]

\item Observe that $\deg\mathfrak{c}\leq t$ and
\[
\mathfrak{c}\left(0\right)=\sum_{1\leq i\leq n}r_{i}\cdot\mathfrak{c}_{i}\left(0\right)=\sum_{1\leq i\leq n}r_{i}\cdot a_{i}b_{i}=a\cdot b=c
\]

\item Hence party $i$ computes
\[
c_{i}=\sum_{1\leq\ell\leq n}r_{\ell}\cdot u_{\ell i}=\sum_{1\leq\ell\leq n}r_{\ell}\cdot\mathfrak{c}_{\ell}\left(P_{i}\right)=\mathfrak{c}\left(P_{i}\right),
\]
which is a $\left(t+1,n\right)$-SSS share $c_{i}$ of $c=a\cdot b$
\end{itemize}
\end{frame}

\begin{frame}{Privacy}

\begin{itemize}
\item Party $i$ only deals with their respective shares
\item After reconstruction party $i$ has share $c_{i}$ of result $c=a+b$
or $c=a\cdot b$
\item Shares belong to a $\left(t+1,n\right)$-SSS, but adversary can only
corrupt up to $t$ parties
\item No information about other parties' input besides what is implied
by their shares and the final result
\end{itemize}
\end{frame}

\begin{frame}{General functions (I)}

\begin{itemize}
\item Using addition and multiplication we can compute more general functions
\item We represent the function as an arithmetic circuit:
\begin{figure}
\begin{centering}
\includegraphics[width=0.3\columnwidth]{figures/arithmetic_circuit}
\par\end{centering}

\protect\caption{A simple arithmetic circuit}
\end{figure}

\end{itemize}
\end{frame}

\begin{frame}{General functions (II)}

\begin{itemize}
\item We can do this without loss of generality (if $f$ is feasible)
\item Well known in the special case of $\mathbb{F}=\left\{ 0,1\right\} $,
so called Boolean circuits
\item Any computable function can be represented using only \textsc{and}
and \textsc{not} gates

\begin{itemize}
\item \textsc{$a$ and $b$} can be represented by $a\cdot b$
\item \textsc{not} $a$ can be represented by $1-a$
\item The computer is a ``proof by example''
\end{itemize}
\item In the general case represent a function $f:\mathbb{F}^{n}\to\mathbb{F}$
by an arithmetic circuit consisting of addition and multiplication
gates
\item Calculations proceed gate by gate
\end{itemize}
\end{frame}

\begin{frame}{The protocol}


To compute a function $y=f\left(x_{1},\ldots,x_{n}\right)$, we represent
it as an arithmetic circuit
\begin{itemize}
\item Each party begins with private input $x_{i}\in\mathbb{F}$ and shares
it using a $\left(t+1,n\right)$-SSS with all participants
\item The calculation proceeds gate by gate, so that at each point all inputs
and intermediate results are shared with a $\left(t+1,n\right)$-SSS
\item From all remaining gates we randomly choose one for which all inputs
are available
\item At the end $P_{i}$ broadcasts its share $y_{i}$ of the final result
$y=f\left(x_{1},\ldots,x_{n}\right)$
\end{itemize}
\end{frame}

\begin{frame}{Correctness and privacy}

\begin{itemize}
\item Correctness follows from correctness of Shamir's scheme and algorithms
for addition and multiplication
\item Privacy follows from the facts that:

\begin{itemize}
\item The adversary was assumed to only corrupt up to $t$ parties
\item All values are shared with a $\left(t+1,n\right)$-scheme, so the
adversary cannot interfere anything about the honest party's inputs
\item The corrupted parties only learn their own inputs and outputs
\end{itemize}
\item \textit{Everybody} learns the final result (fairness)
\end{itemize}
\end{frame}

\begin{frame}{Tightness of bound (I)}

\begin{itemize}
\item What if $t\geq n/2$?
\item Then there is no protocol with \textbf{perfect privacy} and \textbf{perfect
correctness}

\begin{itemize}
\item Assuming correctness, a infinite powerful passive advisor can\\
violate privacy!
\end{itemize}
\item An example:

\begin{itemize}
\item Consider two parties $P_{i}$ with input bit $b_{i}$ ($i=1,2$)
\item They want to securely compute $r=b_{1}\wedge b_{2}$
\item Both have additional randomness $r_{i}\in\left\{ 0,1\right\} ^{\star}$
\end{itemize}
\end{itemize}
\end{frame}

\begin{frame}{Tightness of bound (II)}

\begin{itemize}
\item Both are exchanging messages $m_{ij}$, $j=1,\ldots,N$.\\
Define the \textbf{transcript}
\[
\mathscr{T}=\left(m_{11},m_{21},m_{12},m_{22},\ldots,m_{1N},m_{2N},r\right)
\]

\item Let $\mathscr{T}\left(b_{1},b_{2}\right)$ be the set of transcripts
for given $b_{1}$ and $b_{2}$
\item Can then show that
\[
\mathscr{T}\left(0,0\right)\cap\mathscr{T}\left(0,1\right)=\textrm{�}
\]

\item Hence if $b_{1}=0$ party $P_{1}$ can check if $\mathscr{T}\in\mathscr{T}\left(0,0\right)$
or $\mathscr{T}\in\mathscr{T}\left(0,1\right)$ and deduce $b_{2}$

\begin{itemize}
\item Might be unfeasible in the real world
\end{itemize}
\end{itemize}
\end{frame}

\subsection{Active secure protocol}
\begin{frame}{Active adversaries}

\begin{itemize}
\item We now want to deal with an active adversary

\begin{itemize}
\item We assume that $t<n/2$ for this part
\end{itemize}
\item In the presence of an active adversary a broadcast (BC) channel cannot
be taken for granted

\begin{itemize}
\item Corrupted party might send different things to different parties
\item However, in the case discussed here there are effective protocols
to emulate a BC
\end{itemize}
\item Corrupted parties can now:

\begin{itemize}
\item Deviate from the protocol
\item Give wrong inputs
\item Might even refuse to respond
\end{itemize}
\end{itemize}
\end{frame}

\begin{frame}{Verifiable secret sharing schemes}

\begin{itemize}
\item We need a \textbf{Verifiable Secret Sharing (VSS)} scheme

\begin{itemize}
\item A VSS is a SSS, that allows the parties to verify that they have consistent
shares
\end{itemize}
\item We implement the active secure protocol by emulating the previous
protocol and:

\begin{itemize}
\item We make all \textbf{parties committed to their respective shares}
\item We ensure that all shares are computed correctly
\end{itemize}
\end{itemize}
\end{frame}

\begin{frame}{Modeling security}

\begin{itemize}
\item In the ideal world all a corrupted party can do is specify an alternative
input $x'_{i}$ for $\textrm{\ensuremath{\mathcal{F}_{\mathrm{SFE}}^{f}}}$
\item We require that all deviations of the protocol can be modeled by choosing
alternative inputs
\item What if a corrupted party refuses to give any input?

\begin{itemize}
\item The protocol can potentially deadlock
\item Possible solution: other parties simulate this party with input $x_{i}=0$
\end{itemize}
\end{itemize}
\end{frame}

\begin{frame}{Commitments}

\begin{itemize}
\item How can a party $P_{i}$ commit to a value $a\in\mathbb{F}$?
\item To model this we introduce another ideal functional $\mathcal{F}_{\mathrm{COM}}$

\begin{itemize}
\item In the real world this will be implemented collectively by the other
parties
\end{itemize}
\item Can then commit to $a$ and access $a$ via $\mathcal{F}_{\mathrm{COM}}$
using a interface with given commands
\end{itemize}
\end{frame}

\begin{frame}{Interface of $\mathcal{F}_{\mathrm{COM}}$}

\begin{itemize}
\item We now define an interface for $\mathcal{F}_{\mathrm{COM}}$ consisting
of commands
\item For execution \textit{all honest parties have to agree on the command}
send to $\mathcal{F}_{\mathrm{COM}}$ as the implementation will require
them to actively participate
\item Basic commands for committing and revealing of values

\begin{itemize}
\item Values committed to not known by other parties than the committer
\end{itemize}
\item Also implementing manipulation commands

\begin{itemize}
\item Add and multiply committed shares
\item Allows us to eventually to emulate $\textrm{\ensuremath{\mathcal{F}_{\mathrm{SFE}}^{f}}}$
by $\mathcal{F}_{\mathrm{COM}}$ 
\end{itemize}
\end{itemize}
\end{frame}

\begin{frame}{Basic commands of $\mathcal{F}_{\mathrm{COM}}$}

\begin{itemize}
\item \texttt{\textbf{\textsc{commit}}} of $P_{i}$ to $a\in\mathbb{F}$,
denoted by $P_{i}:\left[a\right]_{i}\Leftarrow a$

\begin{itemize}
\item After successful execution $\mathcal{F}_{\mathrm{COM}}$ stores $\left(P_{i},a\right)$
\end{itemize}
\item \texttt{\textbf{\textsc{public commit}}} of all parties to $a\in\mathbb{F}$

\begin{itemize}
\item By $\left[a\right]_{i}\Leftarrow a$ we denote the use of the \texttt{\textbf{\textsc{public
commit}}} command to force $P_{i}$ to commit to $a$
\item $\mathcal{F}_{\mathrm{COM}}$ stores $a$ for all $P_{i}$
\end{itemize}
\item \texttt{\textbf{\textsc{open}}} of $a\in\mathbb{F}$ to all parties
assuming some $\left[a\right]_{i}$ is stored, denoted by $a\Leftarrow\left[a\right]_{i}$ 

\begin{itemize}
\item All parties learn $a$
\end{itemize}
\item \texttt{\textbf{\textsc{designated open}}} of $a\in\mathbb{F}$ to
party $P_{j}$ assuming some $\left[a\right]_{i}$ is stored, denoted
by $P_{j}:a\Leftarrow\left[a\right]_{i}$ 

\begin{itemize}
\item Party $P_{j}$ learns $a$
\end{itemize}
\end{itemize}
\end{frame}

\begin{frame}{Manipulation commands of $\mathcal{F}_{\mathrm{COM}}$}

\begin{itemize}
\item \texttt{\textbf{\textsc{add}}} of two values $a,b\in\mathbb{F}$,
assuming some $\left[a\right]_{i}$ and $\left[b\right]_{i}$ is stored

\begin{itemize}
\item Denoted by $\left[a+b\right]_{i}\leftarrow\left[a\right]_{i}+\left[b\right]_{i}$
\end{itemize}
\item \texttt{\textbf{\textsc{multiplication by a constant}}} of $a\in\mathbb{F}$
with an $\alpha\in\mathbb{F}$, assuming some $\left[a\right]_{i}$
is stored

\begin{itemize}
\item Denoted by $\left[\alpha a\right]_{i}\leftarrow\alpha\left[a\right]_{i}$
\end{itemize}
\item \texttt{\textbf{\textsc{transfer}}} of $a\in\mathbb{F}$ to all parties
assuming some $\left[a\right]_{i}$ is stored

\begin{itemize}
\item $P_{j}$ learns $a$ and commits to it, denoted by $\left[a\right]_{j}\leftarrow\left[a\right]_{i}$
\end{itemize}
\item \texttt{\textbf{\textsc{multiplication}}} of two values $a,b\in\mathbb{F}$,
assuming some $\left[a\right]_{i}$ and $\left[b\right]_{i}$ is stored

\begin{itemize}
\item Denoted by $\left[a\cdot b\right]_{i}\leftarrow\left[a\right]_{i}\cdot\left[b\right]_{i}$
\end{itemize}
\end{itemize}
\end{frame}

\begin{frame}{Implementation}

\begin{itemize}
\item The \texttt{\textbf{\textsc{transfer}}} and \texttt{\textbf{\textsc{multiplication}}}
commands are high level commands

\begin{itemize}
\item Can be implemented using the other commands
\end{itemize}
\item As an example we show how to implement the \texttt{\textbf{\textsc{multiplication}}}
command
\item For brevity we omit here the implementation of the \texttt{\textbf{\textsc{transfer}}}
command

\begin{itemize}
\item Details can be found in \cite{CDN09}
\end{itemize}
\end{itemize}
\end{frame}

\begin{frame}{The \texttt{\textbf{\textsc{multiplication}}} command}


We implement $\left[a\cdot b\right]_{i}\leftarrow\left[a\right]_{i}\cdot\left[b\right]_{i}$
\begin{enumerate}
\item $P_{i}:\left[c\right]_{i}\Leftarrow a\cdot b$ ($P_{i}$ knows $a$
and $b$)
\end{enumerate}

If $P_{i}$ is honest, $c$ will be correct. But $P_{i}$ might cheat.
Hence every $P_{k}$ carries out the following consistency check:
\begin{enumerate}
\item $P_{i}$ chooses $\alpha\in\mathbb{F}$ uniform at random
\item $P_{i}:\left[\alpha\right]_{i}\Leftarrow\alpha$ (\texttt{\textbf{\textsc{commit}}}
to $\alpha$)
\item $P_{i}:\left[\gamma\right]_{i}\Leftarrow\alpha b$ (\texttt{\textbf{\textsc{commit}}}
to $\alpha b$)
\item $P_{k}$ broadcasts a \textbf{challenge} $e\in\mathbb{F}$ uniform
at random
\item $\left[A\right]_{i}\leftarrow e\left[a\right]_{i}+\left[\alpha\right]_{i}$;
$A\Leftarrow\left[A\right]_{i}$ (\texttt{\textbf{\textsc{open}}}
$A$)
\item $\left[D\right]_{i}\leftarrow A\left[b\right]_{i}-e\left[c\right]_{i}-\left[\gamma\right]_{i}$;
$D\Leftarrow\left[D\right]_{i}$ (\texttt{\textbf{\textsc{open}}}
$D$)
\item The proof is accepted if $D=0$
\end{enumerate}
\end{frame}

\begin{frame}{The \texttt{\textbf{\textsc{multiplication}}} command}

\begin{itemize}
\item If $\left[c\right]_{i}=\left[a\right]_{i}\cdot\left[b\right]_{i}$
then $D=0$.
\item If $P_{i}$ cheated and committed to a $c=a\cdot b+\Delta$,\\
then $D\neq0$ with probability $\left|\mathbb{F}\right|^{-1}$

\begin{itemize}
\item As \textrm{$\mathbb{F}$ is finite, }$D=0$ can still happen coincidentally
\item There are $n-t>n/2$ honest parties, so probability that all proofs
of honest parties are accepted in the case of cheating is $\leq\left|\mathbb{F}\right|^{-n/2}$
\item By repeating the proof several times, probability can be further reduced
\end{itemize}
\item We now present the actual protocol using $\mathcal{F}_{\mathrm{COM}}$ 
\end{itemize}
\end{frame}

\begin{frame}{The active secure protocol (I)}


\textbf{Input sharing}
\begin{itemize}
\item Party $P_{i}$ holds input $x_{i}$ and shares it using Shamir's scheme
\item Ensure correct shares and that parties are committed to their shares
\end{itemize}

Protocol
\begin{enumerate}
\item $P_{i}:\left[x_{i}\right]_{i}\Leftarrow x_{i}$ (\texttt{\textbf{\textsc{commit}}}
to input)
\item $P_{i}$ chooses a polynomial $\mathscr{P}_{i}\left(z\right)=x_{i}+\sum_{j=1}^{t}\alpha_{j}z^{j}$\\
uniform at random
\item $P_{i}:\left[\alpha_{j}\right]_{i}\Leftarrow\alpha_{j}$, $\forall j$
(\texttt{\textbf{\textsc{commit}}} to coefficients)
\item $P_{i}:\left[\mathscr{P}_{i}\left(P_{\ell}\right)\right]_{i}\leftarrow x_{i}+\sum_{j=1}^{t}\left[\alpha_{j}\right]_{i}P_{\ell}^{j}$,
$\ell=1,\ldots,n$\\
(evaluating the shares for all parties)
\item $\left[\mathscr{P}_{i}\left(P_{\ell}\right)\right]_{\ell}\leftarrow\left[\mathscr{P}_{i}\left(P_{\ell}\right)\right]_{i}$,
$\ell=1,\ldots,n$\\
(\texttt{\textbf{\textsc{transfer}}} of all shares to the respective
parties)
\end{enumerate}
\end{frame}

\begin{frame}{The active secure protocol (II)}


\textbf{Arithmetic operations}
\begin{itemize}
\item Function $f$ was assumed to be represented by an arithmetic circuit
\end{itemize}

Addition
\begin{enumerate}
\item $\left[c\right]_{i}\leftarrow\left[a\right]_{i}+\left[b\right]_{i}$
\end{enumerate}

Multiplication
\begin{enumerate}
\item $\left[d_{i}\right]_{i}\leftarrow\left[a_{i}\right]_{i}\cdot\left[b_{i}\right]_{i}$
(local multiplication)
\item $P_{i}$ shares $\left[d_{i}\right]_{i}$ (like in input sharing part),\\
hence $P_{\ell}$ is committed to $\left[d_{i\ell}\right]_{\ell}$ 
\item $\left[c_{\ell}\right]_{\ell}\leftarrow\sum_{i=1}^{n}r_{i}\left[d_{i\ell}\right]_{\ell}$\\
(recombination with recombination vector $r_{1},\ldots,r_{n}$)
\end{enumerate}
\end{frame}

\begin{frame}{The active secure protocol (III)}


\textbf{Reconstruction}
\begin{itemize}
\item Party $P_{i}$ committed to share $y_{i}$
\end{itemize}

If $P_{j}$ is supposed to learn $y$:
\begin{enumerate}
\item $P_{j}:y_{i}\Leftarrow\left[y_{i}\right]_{i}$, $i=1,\ldots,n$
\end{enumerate}

Note:
\begin{itemize}
\item If share $y_{i}$ is stored in $\mathcal{F}_{\mathrm{COM}}$, it is
consistent
\item If $P_{i}$ cheats, it might be not recorded and the opening fails
\item As there are $n-t>t$ honest parties, still can reconstruct $y$
\end{itemize}
\end{frame}

\begin{frame}{Security of the protocol}

\begin{itemize}
\item The protocol ensures correct and consistent shares at every point
\item However, a corrupted party might refuse to carry out a given command
\item If this happens with party $P_{j}$:

\begin{itemize}
\item Input phase: other parties take input $x_{j}=0$\\
and $0$-polynomial for $P_{j}$
\item Addition cannot fail
\item Multiplication: if $P_{j}$ has been disqualified, open its input
and restart the calculation and openly simulate this party
\item Reconstruction was already discussed
\end{itemize}
\end{itemize}
\end{frame}

\begin{frame}{Implementation of $\mathcal{F}_{\mathrm{COM}}$ (I)}

\begin{itemize}
\item Import question: how to implement the ideal functional $\mathcal{F}_{\mathrm{COM}}$
by a protocol $\pi_{\mathrm{COM}}$?
\item We will emulate it by all (honest) parties
\item Assume information-theoretic scenario with $t<n/3$

\begin{itemize}
\item In the cryptographic scenario can relax to $t<n/2$
\end{itemize}
\item We just give an outline of the realization
\end{itemize}
\end{frame}

\begin{frame}{Implementation of $\mathcal{F}_{\mathrm{COM}}$ (II)}

\begin{itemize}
\item Idea: can implement \texttt{\textbf{\textsc{commit}}} of party $P_{j}$
to $a\in\mathbb{F}$ by using a SSS

\begin{itemize}
\item Then we easy to implement \texttt{\textbf{\textsc{open}}}, \texttt{\textbf{\textsc{add}}}
and \texttt{\textbf{\textsc{multiplication by a constant}}} (and hence
\texttt{\textbf{\textsc{transfer}}} and \texttt{\textbf{\textsc{multiplication)}}}
\item If $P_{j}$ is honest, adversary will not learn $a$
\item But: If $P_{j}$ is corrupted, might give \textbf{inconsistent shares}
\end{itemize}
\item Hence we have to \textbf{force consistent shares}
\end{itemize}
\end{frame}

\begin{frame}{Corrupted shares}

\begin{itemize}
\item Note that if even $<n/3$ shares are corrupted, the secret is still
uniquely defined
\item Consider secret $s\in\mathbb{F}$ shares with polynomial $p$ with
$\deg p\leq t$
\item The shares are
\[
\mathbf{s}=\left(p\left(P_{1}\right),\ldots,p\left(P_{n}\right)\right)
\]

\item Consider an error $\mathbf{e}\in\mathbb{F}^{n}$ with Hamming-weight
$w_{\mathrm{H}}\left(\mathbf{e}\right)\leq t$
\item Then both $\mathbf{s}$ and $\tilde{\mathbf{s}}=\mathbf{s}+\mathbf{e}$
uniquely define the same $s$
\end{itemize}
\end{frame}

\begin{frame}{Forcing consistent shares (I)}

\begin{itemize}
\item In principle can check for consistent shares, if dealer broadcasts
the polynomial

\begin{itemize}
\item But this reveals the secret
\end{itemize}
\end{itemize}

Instead use algorithm:
\begin{enumerate}
\item $P_{j}$ shares secret $s\in\mathbb{F}$ with shares $\left\{ s_{i}\right\} $
\item $P_{j}$ picks a $r\in\mathbb{F}$ uniform at random and shares $\left\{ r_{i}\right\} $
\item A challenge $e\in\mathbb{F}$ is chosen and $\forall i$ party $P_{i}$
computes
\[
a_{i}=e\cdot s_{i}+r_{i}
\]

\item Then make consistency check of $\left\{ a_{i}\right\} $ for value
$a=e\cdot s+r$
\end{enumerate}
\end{frame}

\begin{frame}{Forcing consistent shares (II)}

\begin{itemize}
\item Value $a$ is randomly distributed, revealing is unproblematic
\item If shares $\left\{ s_{i}\right\} $ and $\left\{ r_{i}\right\} $
are consistent, so are the $\left\{ a_{i}\right\} $
\item Otherwise probability that the $\left\{ a_{i}\right\} $ are consistent
by\\
coincident is $\left|\mathbb{F}\right|^{-1}$
\item What if the $\left\{ a_{i}\right\} $ are consistent, but a corrupted
party broadcasts a value $\tilde{a}_{i}\neq a_{i}$?

\begin{itemize}
\item We employ \textbf{dispute control}
\end{itemize}
\end{itemize}
\end{frame}

\begin{frame}{Dispute control (I)}

\begin{itemize}
\item With each party we associate a public \textbf{dispute set} $D_{i}\subseteq\left\{ P_{1},\ldots,P_{n}\right\} $

\begin{itemize}
\item At the beginning $D_{i}=\textrm{�}$
\end{itemize}
\item If some party $P_{j}$ broadcasts an inconsistent share $\tilde{a}_{j}$:

\begin{itemize}
\item $P_{j}$ is added to $D_{i}$ ($P_{i}$ is dealer)
\item If dealer $P_{i}$ is honest $P_{j}\in D_{i}$ means that $P_{j}$
is corrupted
\item Test is repeated with $a_{j}=0$ (\textbf{corrected sharing})
\end{itemize}
\end{itemize}
\end{frame}

\begin{frame}{Dispute control (II)}

\begin{itemize}
\item If in one of the repetitions:

\begin{itemize}
\item $P_{i}$ broadcasts a polynomial with $p\left(P_{j}\right)\ne0$ for
$P_{j}\in D_{i}$
\item $\left|D_{i}\right|>t$ (at least one honest party is in dispute with
$P_{i}$)
\end{itemize}
\item Then: remaining parties accuse dealer $P_{i}$ of being corrupt

\begin{itemize}
\item All messages from $P_{i}$ will be ignored from now on
\end{itemize}
\item Employing dispute control allows us to ensure consistent shares and
to implement $\mathcal{F}_{\mathrm{COM}}$ (which emulates $\mathcal{F}_{\mathrm{SFE}}^{f}$)
\end{itemize}
\end{frame}

\begin{frame}{Conclusions}

\begin{itemize}
\item We showed how to implement MPCs using \textbf{Shamir's Scheme}
\item For a \textbf{passive} threshold adversary we have to require $t<n/2$
\item For an \textbf{active} threshold adversary in the information-theoretic
scenario we need $t<n/3$

\begin{itemize}
\item Protect against active attacks with \textbf{VSS}s and \textbf{dispute
control}
\end{itemize}
\item \textbf{For more information refer to Ref.~\cite{CDN09,HM97}}
\end{itemize}
\end{frame}
\appendix

\section*{Appendix}


\subsection*{For Further Reading}
\begin{frame}{Bibliography}


\bibliographystyle{plain}
\addcontentsline{toc}{section}{\refname}\bibliography{literature}
\end{frame}

\end{document}
